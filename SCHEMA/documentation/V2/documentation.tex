\documentclass[12pt,a4paper,oneside]{book} % ou article, memoir, report, etc.

%usepackage permet d'utiliser un module complémentaire.

% Marges, retraits et espacement
\usepackage[margin=2.5cm]{geometry}
%espacement des lignes
\usepackage{setspace}
\onehalfspacing
\setlength\parindent{1cm}

%Le package Babel permet de gérer différentes normes linguistiques et typographiques
\usepackage[english,italian,french]{babel}


%Nouvelle syntaxe ici: une commande avec une option entre []
\usepackage[utf8]{inputenc}

%Gérer l'encodage des caractères en sortie
\usepackage[T1]{fontenc}

%Changer la fonte de caractère
\usepackage{lmodern}
%\usepackage{charter}
%Pour les maniaques du Times \usepackage{txfonts}
%Palatino \usepackage{mathpazo}

%Métadonnées du document
%Auteur
\author{}
\title{}
%La commande date est optionnelle
%\date{2 janvier 1950}
\date{Version du \today}


%Écrire les mots inconnus du dictionnaire pour la césure/l'hyphénation
%\hyphenation{Ajou-t-ons}


%Autres modules complémentaires
\usepackage{lettrine}
\usepackage[pdftex]{graphicx}

%pour le mode paysage je dois utiliser deux packages
%\usepackage{lscape}%pour le mode paysage
%\usepackage{pdflscape}%pour l'indiquer dans les métadonnées du pdf.

%pour le dessin
\usepackage{tikz}%pour le dessin
\usepackage{tikz-qtree}%pour les arbres 

\usepackage{caption}%pour ne pas numéroter les figures (on ajoute une étoile après caption)

%pour créer un Index
\usepackage{makeidx}
%pour activer la création de l'index.
\makeindex 

%Appeller le package pour les bibliographies
%\usepackage[backend=biber, sorting=nyt, style=inha]{biblatex}
%appel de la ressource bibliographique.
%\addbibresource{•}
%pour les guillemets
\usepackage[babel]{csquotes}

\begin{document}

\part{Documentation TEI}

L'édition électronique de la correspondance d'Armand Horel, est une édition XML-TEI. Cette partie est relative à la documentation concernant l'encodage et la production de fichiers.

La Bibliothèque de documentation internationale contemporaine (BDIC) s'est lancée dans un important programme de numérisation de ses collections, accessibles depuis sa bibliothèque numérique\footnote{voir : http://argonnaute.u-paris10.fr/}. 
Proposer  une édition électronique à partir de ce corpus numérisés, permet de mettre en valeur certains fonds, tout en offrant des outils de recherche que les versions fac-similaires ne permettent pas.

L'édition électronique de la correspondance d'Armand Horel s'inscrit, plus largement, dans un projet visant à offrir, aux chercheurs qui le souhaitent, une solution leur permettant de réaliser ce type d'exercice dans le cadre de leur travaux.

\chapter{Introduction}

Procéder à une édition en ligne nécessite se s'interroger sur le langage à utiliser pour l'encodage du texte. Une publication web repose sur des pages HTML et nécessite donc, \textit{a minima}, un balisage présentationnel de la source. Toutefois, depuis la publication en 2009 du Référentiel Général d'Interopérabilité (RGI) par la Direction générale de modernisation de l'État, c'est la technologie XML (langage de balisage descriptif) qui est favorisée en ce qu'elle assure la pérennisation et interopérabilité de l'information. Parallèlement, le dialecte XML-TEI, dont le champ d'application est maintenant relativement étendu, est devenu un standard concernant l'édition scientifique des sources primaires. C'est donc d'elle même que cette solution s'est imposée pour le développement d'un modèle d'édition de correspondance.  
\bigskip 

Ce choix n'est pas pour autant une solution applicable directement. Le balisage du texte nécessite de réfléchir à l'élaboration d'un schéma, contenant les règles à respecter concernant l'usage de la TEI et qui oblige à procéder avant toute chose à une analyse structurelle et intellectuelle de la source.

\chapter{Langage de balisage et Text encoding initiative (TEI)}

\section{La balisage descriptif}
D'une manière générale, les langages de balisage sont des dialectes informatiques adaptés à l'enrichissement d'informations textuelles. De plus, sa structure dans laquelle une balise correspond à une unité syntaxique délimitant une séquence au sein d'un flux de caractères est compréhensible par la machine.
\bigskip
On dénombre trois catégorie de langage de balisage : 
\begin{itemize}
\item les langages procéduraux dans lesquels les balises correspondent à des instructions exécutables par un programme informatique,
\item les langages présentationnels qui ont pour fonction de mettre en forme le texte,
\item les langages descriptifs qui présentent l'avantage de distinguer le contenu de la forme permettant ainsi d'en séparer le traitement. Ce dernier garantie ainsi une meilleure portabilité des fichiers numériques.
\end{itemize}
\bigskip 

Un langage descriptif réside donc avant tout dans l'analyse de la source textuelle au sein de laquelle il convient d'identifier sa structure sémantique nonobstant tout traitement présentationnel. Il s'agit dont enrichir le document d'informations sémantiques afin de pouvoir proposer par exemple plusieurs lectures d'un document. trois grande étapes sont nécessaires à la réalisation de ce travail:
\bigskip

\begin{itemize}
\item L'analyse sémantique du document : Il s'agit d'identifier les différent éléments qui composent le texte (paragraphe, signature, date, noms de personne\dots
\item Le choix des balises : quelle balise peut-on appliquer à un élément donné.
\item L'encodage. 
\end{itemize} 
\bigskip 

Ce type de langage descriptif repose sur une structure hiérarchique. Comme nous l'avons vu, le balisage repose sur l'identification d'une séquence au sein d'un flux de caractère. C'est hiérarchique parce que les éléments (paragraphes, phrases\dots) sont imbriqués les uns dans les autres et sont donc liés entre eux par une relation linéaire.
\bigskip 

Les langages à balises descriptifs présentent au final un certain nombre d'avantages :

\begin{itemize}
\item La processus d'établissement du texte est simplifié car il ne repose que sur identification du contenu non pas sur sa présentation ou la compréhension du programme. 
\item Le document est indépendant de l'apparence formelle que l'on souhaite lui donner. 
\item Ce sont des langages interopérables qui facilitent le partage de données.
\end{itemize}
\bigskip 

Ce type de langage est tout à l'avantage des éditeurs dans la mesure où ; on limite les risques d'incompatibilité, ce type de fichier permet de proposer plusieurs éditions à partir du même artefact numérique, enfin, il est possible de générer de manière automatique les informations bibliographiques du document évitant ainsi des erreurs ou autorisant leur versement automatique dans des bases de données en ligne.
\bigskip 

En revanche ce choix implique bien souvent un cadre technique de travail moins confortable que l'utilisation d'un traitement de texte que tout un chacun maîtrise plus ou moins.
\bigskip 

La technologie XML (eXtensible markup Language), est un métalangage de balisage structuré, c'est à dire qu'il respecte une structure hiérarchique formant une arborescence. Elle permet le développement de dialectes descriptifs interopérables. En effet, si sa structure est compréhensible par la machine, ce n'est pour autant pas un langage à proprement parlé et ne propose donc pas une véritablement sémantique. XML se contente d'énoncer un ensemble de règles sur ce que doit être un document bien formé et valide ; il nécessite donc d'élaborer, ou de choisir un vocabulaire spécialisé, comme la TEI. 

\section{Text Encoding Initiative (TEI)}

La TEI est un groupement international qui a pour finalité de développer et maintenir un standard pour l'édition de texte sous forme numérique.
Il s'agit donc d'un vocabulaire XML spécialisé dans l'édition des sources primaires. Son champ d'application est maintenant relativement étendue, ouvrages imprimés anciens, textes médiévaux, chartes et documents, cours écrits ou oraux etc. Toutefois, ce que propose la TEI n'est pas tant un schéma général qu'un cadre de développement composé d'un dialecte explicité par une documentation, le tout réuni dans des \textit{Guidelines}.
\bigskip 

A la fins des années 1980, des chercheurs et universitaires, déjà impliqués dans la production de textes sous forme numérique font le constat d'un manque de solution concernant l'échange de texte résultant de leur recherches. En 1987, lors d'une rencontre organisée par l'Association for Computers and Humanities (ACH), au Vassar College de Poughkeepsie, une trentaine de chercheurs et professionnels s'accordent sur le besoin de développer un cadre de pratique commune. Ils aboutissent à la formulation des principes de Poughkeepsie, dont la finalité est l'élaboration de \textit{Guidelines} (recommandations) avec pour objectifs principaux de :
\begin{itemize}
\item Fournir un format d'échange standard de données pour la recherche en Humanités,
\item Proposer des principes d'encodage de texte dans ce même format,
\item Définir une syntaxe et un schéma,
\item Garantir autant que possible la compatibilité avec les standards existants.
\end{itemize}

L'ACH, rejoint par l'Association for Literary and Linguistic Computing et l'Association for Computational Linguistic établirent la Text Encoding Initiative (TEI) afin de mener le projet, dans plusieurs langues et à un niveau international. Les premières \textit{Guidelines} furent publiées en 1993.
Basé à l'origine sur la technologie SGML (Standard Generalized Markup Language), la TEI embrassera XML dès sa création en 1996. Nous sommes actuellement à la cinquièeme version de la TEI dénommée P5 ; elle présente maintenant un certaine nombre d'avantage : 

\begin{itemize}
\item Adaptabilité à toute forme de document
\item Expressivité de part la granularité qu'elle propose
\item C'est une standard internationnale assurant interopérabilité et pérennité. 
\item Elle est basé sur la technologie XML,
\item Particulièrement adapté à l'édition électronique.
\end{itemize}

\subsection{La production de source primaire}

Plus que la publication d'un standard, la TEI émet des recommandations relativement simples et compréhensibles sur les conventions d'encodage à adopter. De plus sa structure "modulaire" lui permet de répondre à différentes problématiques particulières. Si l'on ajoute le fait, comme nous l'avons vu, que la TEI repose sur la syntaxe XML standardisée par le World Wide Web Consortium (W3C) on comprend qu'elle se soit rapidement imposée comme un standard pour l'édition électronique de sources primaires.

\subsubsection{Un schéma généraliste}
La TEI se veut suffisamment riche pour être adaptée à la multitude des champs des humanités. Cependant, si l'objectif affiché de proposer un standard est censé assurer interopérabilité et pérennité, la généricité de ce dernier, oblige à faire des choix devenant par la même occasion un frein à ces objectifs. Ces choix sont nécessaires pour cadrer au maximum une pratique d'encodage et permettre ainsi un traitement automatisé des fichier. Ainsi l'encodage TEI, n'est pas un fin en soi et il ne suffit donc pas qu'un texte soit encodé avec ce langage pour assurer sa compatibilité. Les documents doivent être avant issue d'une même logique éditoriale.

\subsubsection{Lecture du texte}
  
%revoir cette partie.
Une modélisation répond avant tout à un besoin éditorial et à la représentation que l'on veut donner du texte parmi la multitude de lecture possible. La TEI, parce qu'elle est basée sur la structure hiérarchique de XML, impose une certaine vision du texte. Cette syntaxe XML permet de répondre aux besoins les plus généraux tout en facilitant par la suite le traitement et la transformation des éléments TEI en éléments HTML. Dans d'autres cas l'arborescence imposé par le modèle XML, peut sembler inadaptée voir lourde car elle impose tout de même une interprétation éditoriale de la source. 
Bien qu'imparfaite cette solution reste dans l'immense majorité des cas acceptable et suffisante ; de plus elle permet de bénéficier des puissants outils XML. 
C'est donc de l'interprétation du texte qu'il est question. Il faut bien avoir à l'esprit que la TEI, bien que considérée comme le standard d'édition dans le domaine académique, n'est pas pour autant une solution générique pouvant répondre à tous les cas de figures, une modélisation, même si elle peut répondre à plusieurs problématiques, propose déjà une interprétation du texte et donc une lecture, "reflet des des questionnements à l'œuvre au moment de l'acte d'édition".
%reprendre cette phrase.   




\chapter{Modélisation et TEI}
\section{Modélisation TEI et personnalisation}
\subsection{Modélisation}
Ce que propose la TEI n'est pas tant un schéma qu'un cadre de développement. Tout d'abord parce qu'elle procure à l'éditeur un ensemble de recommandation qu'il est parfois nécessaire d'adapter aux spécificité du projet d'édition. Ensuite parce qu'elle adopte une organisation "modulaire". Chaque module répondant plus ou moins à un type de problématique. C'est ce qui permet à la TEI d'être particulièrement flexible, même si cela peut dérouter l'encodeur inexpérimenté. Il existe souvent plusieurs solutions d'encodage pour un objet donné. Ce choix doit alors être pris en ayant à l'esprit tous les "tenants et aboutissants, impliquant des retours incessants vers les recommandations de la TEI"

Ceci explique cette nécessité de personnaliser le modèle TEI afin de l'adapter au mieux aux spécificités du texte que l'on encode. On imagine aisément que les besoins ne sont pas les mêmes pour l'encodage d'un texte médiéval et celui d'un correspondance contemporaine.
\bigskip 

\subsection{Personnalisation}
\subsubsection{Réalisation d'un schéma}
La TEI a été imaginée comme un ensemble de module, que l'on peut assembler afin de répondre aux spécificités d'un projet. Toute modélisation est établie par un schéma correspondant donc à un sous ensemble de la TEI et répondant aux besoins de son projet. Ce schéma permet à la fois de contrôler la production d'un fichier, mais aussi de valider le contenu d'un fichier en associant ce dernier au schéma.
\bigskip 

Un schéma est généralement établi à partir d'un échantillon représentatif du corpus à encodé et il est susceptible d'évoluer au fur et à mesure de l'avancée de travaux.
\bigskip 

Chaque module est documenté dans une chapitre des \textit{Guidelines} et il définie un certaine nombre des composantes du modèle (éléments, classes, macros).
 


 En revanche ce n'est pas tant un schéma qui est proposé qu'un cadre de développement. La TEI adopte une organisation modulaire. Chaque module répond à une problématique spécifique en définissant un certain nombre des composantes d'une modélisation (éléments, attributs, classes. . .). Ils sont documentés par un chapitre dans les Guidelines\footnote{Recommandation TEI} et chacun est libre des les inclure ou non, en fonction de ses besoins dans sa personnalisation.






\end{document}